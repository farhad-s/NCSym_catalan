\documentclass[11pt]{amsart}

\usepackage[usenames,dvipsnames,svgnames,table]{xcolor}
\usepackage[colorlinks=true, pdfstartview=FitV, linkcolor=blue, citecolor=blue, urlcolor=blue]{hyperref}

\usepackage[centering]{geometry}                % See geometry.pdf to learn the layout options. There are lots.
\geometry{letterpaper}                   % ... or a4paper or a5paper or ...
%\geometry{landscape}                % Activate for for rotated page geometry
%\usepackage[parfill]{parskip}    % Activate to begin paragraphs with an empty line rather than an indent
\usepackage{graphicx}
\usepackage{amssymb}
\usepackage{epstopdf}
\usepackage{lscape}
\usepackage{xcolor}
\usepackage[utf8]{inputenc}
\usepackage{tikz,caption}
\DeclareGraphicsRule{.tif}{png}{.png}{`convert #1 `dirname #1`/`basename #1 .tif`.png}
\usepackage{enumitem}
\setlist[itemize]{leftmargin=2em}
\setlist[enumerate]{leftmargin=2em}
\usepackage{booktabs}
\usepackage{multirow}
\usepackage{mathtools}
\usepackage{mathrsfs}
\usepackage[aligntableaux=center, smalltableaux]{ytableau}
\usepackage{youngtab}

\newtheorem{thm}{Theorem}

\definecolor{darkblue}{rgb}{0.0,0,0.7} % darkblue color
\definecolor{darkred}{rgb}{0.7,0,0} % darkred color
\definecolor{darkgreen}{rgb}{0, .6, 0} % darkgreen color

% Dark red emphasis
\newcommand{\defncolor}{\color{darkred}}
\newcommand{\defn}[1]{{\defncolor\emph{#1}}} % emphasis of a definition

\newtheorem{theorem}{Theorem}[section]
\newtheorem{proposition}[theorem]{Proposition}
\newtheorem{corollary}[theorem]{Corollary}
\newtheorem{lemma}[theorem]{Lemma}
\newtheorem{conjecture}[theorem]{Conjecture}
\theoremstyle{definition}
\newtheorem{definition}[theorem]{Definition}
\newtheorem{example}[theorem]{Example}
\newtheorem{remark}[theorem]{Remark}
\numberwithin{equation}{section}

\def\NN{{\mathbb N}}
\def\CC{{\mathbb C}}
\def\ZZ{{\mathbb Z}}

\usepackage[colorinlistoftodos]{todonotes}

\newcommand{\mike}[1]{\todo[size=\tiny,color=green!30]{#1 \\ \hfill --- Mike}}
\newcommand{\lucas}[1]{\todo[size=\tiny,color=red]{#1 \\ \hfill --- Lucas}}
\newcommand{\felix}[1]{\todo[size=\tiny,color=Cyan]{#1 \\ \hfill --- Felix}}
\newcommand{\farhad}[1]{\todo[size=\tiny,color=blue!50]{#1 \\ \hfill --- Farhad}}
\newcommand{\eric}[1]{\todo[size=\tiny,color=BurntOrange!50]{#1 \\ \hfill --- Eric}}
\newcommand{\nicolas}[1]{\todo[size=\tiny,color=purple!50]{#1 \\ \hfill --- Nicolas}}
\setlength{\marginparwidth}{28mm}


\title{Catalan Hopf sub-algebras of $\mathsf{NCSym}$}
\author{}

\begin{document}
\maketitle

\section{Project description}

\noindent
To identify all the Hopf sub-algebras of $\mathsf{NCSym}$ that are of dimension Catalan and classify them.


\begin{itemize}
\item Introduction - notation and background
\cite{AT20, BBT14, BHRZ05, B08, BZ09, F12, LM11, NT05, RS06}
\begin{itemize}
\item Set partitions, non-crossing and non-nesting
\item $\mathsf{NCSym}$ as a Hopf algebra of set partitions which is non-commutative but co-commutative.
\item (Relevant/useful) bases and known formulae for the product and coproduct.
\item subalgebras characterized by primitives
\end{itemize}
\item I think that we have identified all of the possible Hopf subalgebras of $\mathsf{NCSym}$
(or at least the ones that are characterized by their basis of primitives, which I think is all of them).
Let $b_i$ be the number of atomic set partitions of size $i$, for each sequence of integers $(\mathbf{a}) = (a_1, a_2, a_3,\ldots)$ such that
$0 \leq a_i \leq b_i$ there is a Hopf subalgebra $H^{(\mathbf{a})}$.
Let $H^{(\mathbf{a})}_n$ be the graded component of degree $n$,
$$\mathrm{dim}~H^{(\mathbf{a})}_n = \hbox{coefficient of }t^n\hbox{ in }\frac{1}{1 - a_1 t - a_2 t^2 - a_3 t^3 - \cdots}~.$$
\end{itemize}

\section{Notation}

A \defn{set partition} $A$ of a finite set $X$ is a set of sets $A = \{ A_1, A_2, \ldots, A_k \}$ where $A_i \subseteq X$,
$A_i \cap A_j = \emptyset$ if $i \neq j$ and $\cup_{i=1}^k A_i = X$.  The sets $A_i$ are known as the \defn{parts}
of the set partition.  We indicate that $A$ is a set partition of
$[n] := \{1,2, \ldots, n\}$ by the notation $A \vdash [n]$ and the \defn{length} of the
set partition $A$ is $k$ and is denoted $\ell(A)$.

A set partition $A$ is called \defn{crossing} if there exists $A_i \neq A_j$ which
are parts of $A$ with $a < b < c < d$ and $a,c \in A_i$ and $b,d \in A_j$.  A set partition is said to be \defn{noncrossing} otherwise.
A set partition $A$ is said to be \defn{nesting} if there exists $A_i \neq A_j$ which
are parts of $A$ with $a < b < c < d$ and $a,d \in A_i$ and $b,c \in A_j$ and is \defn{nonnesting} otherwise.
\mike{The definition of non-nesting is probably not correct}

We will define a few combinatorial operations on set partitions that we will need to describe the algebra structure.
For $A \vdash [n]$ and $k \in \mathbb{Z}_{\geq 0}$, let $A_{\uparrow k}$ be the set partition
$\{ \{ a + k : a \in A_i \} : 1 \leq i \leq \ell(A) \}$.  If $A \vdash [n]$ is not equal to $B \cup C_{\uparrow k}$
for some set partitions $B \vdash [k]$ and $C \vdash [n-k]$ with $0 < k < n$, then we say that
$A$ is \defn{atomic}.

The algebra of symmetric functions in noncommuting variables was first considered by Wolf \cite{W36}
and much later in a more combinatorial setting by Rosas and Sagan \cite{RS06}.
We will define $\mathsf{NCSym}$ by its Hopf algebra structure as the linear span of set partitions.
We define the finite dimensional vector space
$\mathsf{NCSym}_n = \mathrm{span}_{\mathbb C}\{ \mathbf{h}_A : A \vdash [n] \}$
and set $\mathsf{NCSym} := \bigoplus_{n \geq 0} \mathsf{NCSym}_n$.

The product on $\mathsf{NCSym}$ is defined on the basis elements and extended linearly.
Let $A \vdash [k]$ and $B \vdash [n]$, then
$$\mathbf{h}_A \cdot \mathbf{h}_B = \mathbf{h}_{A \cup B_{\uparrow k}}~.$$

The coproduct on the Hopf algebra $\mathsf{NCSym}$ appears in \cite{B08}
and is equal to
$$\Delta(\mathbf{h}_A) = \sum_{A \subseteq [n]}
\mathbf{h}_{\mathrm{st}(A|_S)} \otimes \mathbf{h}_{\mathrm{st}(A|_{S^c})}~.$$
\mike{notation to be defined above: $A|_S$, $\mathrm{st}(A)$}

\section{Classification of free cocommutative graded connected Hopf algebras}

Given a sequence $\vec{a} = (a_{1}, a_{2}, \ldots)$ of nonnegative integers, let $\mathfrak{L}(\vec{a})$ denote the free, graded Lie algebra with $a_{i}$ generators in degree $i$, and zero generators in degree zero.  The universal enveloping algebra
\[
\mathcal{U}\big(\mathfrak{L}(\vec{a})\big)
\]
is a free, cocommutative, graded, connected Hopf algebra with Hilbert series\lucas{We have a reference for this in Jacobson's book (something by Witt), but this also seems simple enough to sketch in a few sentences, so maybe we should.}
\[
\frac{1}{1 - a_{1}t - a_{2}t^{2} - a_{3}t^{3} - \cdots}.
\]


\begin{thm}[Various Sources]
Let $H$ be a free, cocommutative, graded, connected Hopf algebra, and let $\vec{a} = (a_{1}, a_{2}, \ldots, )$ count the number of free generators of $H$ in each degree.  
\begin{enumerate}
\item There is a graded isomorphism $H \cong \mathcal{U}\big(\mathfrak{L}(\vec{a})\big)$ of Hopf algebras.

\item Every sub-Hopf algebra of $H$ is isomorphic to $\mathcal{U}\big(\mathfrak{L}(\vec{b})\big)$ for some sequence $\vec{b} = (b_{1}, b_{2}, \ldots)$ with $b_{i} \le a_{i}$.

\item For every sequence $\vec{b} = (b_{1}, b_{2}, \ldots)$ with $b_{i} \le a_{i}$, $H$ has a sub-Hopf algebra isomorphic to $\mathcal{U}\big(\mathfrak{L}(\vec{b})\big)$.

\end{enumerate}
\end{thm}
\begin{proof}
Here is a sketch of what we need to formalize:
\begin{enumerate}
\item Using Aliniaeifard--Thiem's result, we only beed to compare the Hilbert series of $H$ and $\mathcal{U}\big(\mathfrak{L}(\vec{a})\big)$ to prove this part.  \lucas{Should we write out their proof to convince ourselves it is true.  It is short.}

\item Let $G \subseteq H$ be a sub-Hopf algebra.  By (1), the Lie algebra of primitives $\mathcal{P}(H) \cong \mathcal{L}(\vec{a})$.  Thus $\mathcal{P}(G)$ is isomorphic to a subalgebra of $\mathcal{L}(\vec{a})$.  

By~\cite{}, $\mathcal{P}(G)$ is a free Lie algebra, so there exists a sequence $\vec{b} = (b_{1}, b_{2}, \ldots)$ for which $\mathcal{P}(G) \cong \mathfrak{L}(\vec{b})$.\lucas{Detail: show that this result plays well with the grading?}  By Milnor--Moore, $G \cong \mathcal{U}\big(\mathfrak{L}(\vec{b})\big)$.

Missing: what can we say about $\vec{b}$, i.e. the rank of $\mathcal{P}(G)$?  I really think it should satisfy $b_{i} \le a_{i}$, but free objects are weird so this needs a proof\lucas{Big detail!!}.  If we are wrong here, we still could have a classification of sub-Hopf algebras by sequences $\vec{b}$, but we would need to figure out the correct constraints on $\vec{b}$.

\item Here, we construct $\mathfrak{L}(\vec{b})$ as a subalgebra of $\mathfrak{L}(\vec{a})$ by taking $b_{i}$ of the free generators in degree $i$.
\end{enumerate}
\end{proof}



%%%%%%%%%%%%%%%%%%%%%%%%%%%%%%%%%%%%%%%%%%%%%%%%%%%
\bibliographystyle{plain}
\bibliography{bibliography}{}

\end{document}
